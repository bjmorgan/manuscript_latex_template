\documentclass[aps,prmaterials,twocolumn,superscriptaddress,reprint,citeautoscript]{revtex4-2}

\usepackage{graphicx}% Include figure files
\usepackage{dcolumn}% Align table columns on decimal point
  \newcolumntype{d}[1]{D{.}{.}{#1}}
\usepackage{bm}% bold math
\usepackage{natbib}
\usepackage[hidelinks]{hyperref}% add hypertext capabilities
\usepackage{orcidlink}
\usepackage{amssymb}
\usepackage[version=4]{mhchem} % Chemical formulae using \ce{}
\usepackage{siunitx}
  \sisetup{per-mode=reciprocal, bracket-unit-denominator=true, sticky-per, uncertainty-mode=separate}%
  \DeclareSIUnit{\Angstrom}{\textup{\AA}}
\usepackage[capitalise]{cleveref}

\graphicspath{{Figures/}} % Path to figures.

% User commands
  \newcommand{\cis}{\emph{cis}}
  \newcommand{\trans}{\emph{trans}}
  \newcommand{\todo}[1]{\color{red}[#1]\color{black}}
  \newcommand{\tocite}{[\color{red}?\color{black}]}
  \newcommand{\highlight}[1]{\color{green}#1\color{black}}
  \newcommand{\sg}{$P$m$\bar{3}$m}
  \newcommand{\note}[1]{\color{red}#1\color{black}}

\begin{document}


\title{Title}

\author{Benjamin J. Morgan \orcidlink{0000-0002-3056-8233}}
\email{b.j.morgan@bath.ac.uk}
\affiliation{Department of Chemistry, University of Bath, Claverton Down BA2 7AY, United Kingdom}
\affiliation{The Faraday Institution, Quad One, Harwell Science and Innovation Campus, Didcot OX11 0RA, United Kingdom}

\date{\today}

\begin{abstract}
  The properties of heteroanionic materials depend on the crystallographic arrangement of their constituent anions. 
  For anion-disordered oxyfluorides, conventional diffraction methods are not able to fully resolve the anion structure because of the poor scattering contrast between oxygen and fluorine, making alternative structure determination methods necessary. 
  We have preformed a joint experimental and computational study of anion structure in cubic \ce{TiOF2} that integrates experimental X-ray PDF and $^{19}$F MAS NMR analysis with DFT, cluster expansion, and genetic algorithm structure-prediction calculations. 
  Our computational modelling predicts that cubic \ce{TiOF2} exhibits strong short-range anion ordering, characterised by preferential \cis-Ti[\ce{O2F4}] coordination. 
  This short-range ordering gives correlated disorder at longer range, consistent with the long-range O/F disorder reported for previous diffraction studies.
\end{abstract}
\keywords{keyword1, keyword2}
\maketitle

%%%%%%%%%%%%%%%%%%%%%%%%%%%%%%%%%%%%%%%%%%%%%%%%%%%%%%%%%%%%%%%%%%%%%
%% Start the main part of the manuscript here.
%%%%%%%%%%%%%%%%%%%%%%%%%%%%%%%%%%%%%%%%%%%%%%%%%%%%%%%%%%%%%%%%%%%%%
\section{Introduction}
Heteroanionic materials, which contain two or more distinct anionic species, offer a rich chemical space for the design and synthesis of materials with tailored properties \cite{HaradaEtAl_AdvMater2019,KageyamaEtAl_NatCommun2018,CharlesEtAl_ChemMater2018}.
Heteroanionic materials exhibit compositional and structural degrees of freedom that are not present in analogous homoanionic materials, and by modulating the relative stoichiometries or arrangements of the member anion species, the properties of heteroanionic materials may be tuned for specific target applications~\cite{WagnerEtAl_PhysRevB2019}.
This rich chemistry and diverse range of corresponding material properties mean that heteroanionic materials find use in a range of critical technologies, including thermoelectrics \cite{ZhaoEtAl_ApplPhysLett2010}, photocatalysis \cite{GouEtAl_ChemMater2020}, and energy storage~\cite{Morgan_ChemMater2021,McCollEtAl_NatureCommun2022,ColesEtAl_JMaterChemA2023}. 

The properties of heteroanionic materials depend on the identities and relative stoichiometries of their constituent anions and on the arrangement of these anions within the host crystal structure.
If the distinct anion species are arranged in a regular, repeating pattern, the anion substructure is ordered and can be fully described within a single unit cell.
If, instead, different anion species jointly occupy the same crystallographic sites, the structure is disordered: anion site occupations are uncorrelated at long range and the anion sub-structure cannot be fully described within a single unit cell.

\begin{figure}[htb!]
  \centering
  \resizebox{8.6cm}{!}{\includegraphics*{example_figure.pdf}} %
  \caption{\label{fig:example_figure}Example figure.}
\end{figure}

\section{Methods}
\cref{fig:example_figure}

\section{Results and Discussion}

\section{Summary and Conclusions}

\section{Associated Content}
\subsection{Supporting Information}

\subsection{Supporting Data}

\section{Acknowledgements}
B.\ J.\ M.\ acknowledges support from the Royal Society (Grant Nos.\ UF130329 and URF\textbackslash R\textbackslash 191006).
We are grateful to the UK Materials and Molecular Modeling Hub for computational resources, which is partially funded by EPSRC (EP/P020194/1).

\bibliography{bibliography}

\end{document}
